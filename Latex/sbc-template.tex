\documentclass[12pt]{article}

\usepackage{sbc-template}

\usepackage{graphicx,url}

% For Brazilian portuguese language
\usepackage[utf8]{inputenc}
\usepackage[brazil]{babel}
     
\sloppy

\title{Aplicativo de Troca de Mensagens Utilizando Criptografia Baseada em
Imagens}

\author{Marcus Antonio Grécia Brandt\inst{1}, Mikhail Yasha Ramalho \inst{1}}

\address{
Centro de Ensino Superior FUCAPI (CESF)\\Av. Gov. Danilo de Mattos Areosa, 381 -- Distrito Industrial -- Manaus -- AM -- Brasil
\email{\{marcusbrandt, mikhailramalho\}@gmail.com}
}

\begin{document} 

\maketitle

\begin{abstract}
  This meta-paper describes the style to be used in articles and short papers
  for SBC conferences. For papers in English, you should add just an abstract
  while for the papers in Portuguese, we also ask for an abstract in
  Portuguese (``resumo''). In both cases, abstracts should not have more than
  10 lines and must be in the first page of the paper.
\end{abstract}
     
\begin{resumo} 
  Aplicações de troca de mensagens possuem naturalmente um forte apelo popular
  e figuram como aplicações de destaque nas lojas de aplicativos. Porém levantam
  questões relativas a segurança e confiabilidade das informações que por eles
  trafegam. Este artigo descreve a implementação de um aplicativo de comunicação
  por meio do bluetooth que utiliza um algoritmo de criptografia baseado em
  imagens como chave de acesso.
\end{resumo}

\section{Introdução}

\section{Fundamentação Teórica} \label{sec:firstpage}

\subsection{ASCII}
O ASCII \textit{American Standard Code for Information Interchange} (Código Padrão Americano Para o Intercâmbio de Infromação) é um esquema de codificação inicialmente baseado no alfabeto americano, posteriormente se tornou um esforço para padronização da representação de caracteres pelos fabricantes de computadores.
\\Cada caracter (pontuação, valores alfanuméricos e valores de controle) é representado por um valor numérico divididos em uma tabela. Originalmente como eram representados apenas caracters americanos, acentuações e letras com essa característica não podiam ser representados, posteriormente a tabela passou por uma revisão e foram criadas 12 novas partes pra suprir essa necessidade.
\subsection{Imagem}
Quando digitalizadas as imagens tambem são representadas por números,  
\subsection{Criptografia}
\subsubsection{Simétrica}
\subsubsection{Assimétrica}
\subsection{Bluetooth}

\section{Sistema}
\subsection{Android}
O protótipo apresentado neste trabalho foi desenvolvido em Android, por ser uma plataforma popular para desenvolvimento de aplicativos móveis. O Android possui interfaces de programação de aplicativo (APIs), que vêm com o seu Software Development Kit (SDK) e possui recursos completos de interface\cite{android:2014:Online}
Para este protótipo utilizamos a versão 4.4.2 do Android conhecida como Jelly Beans, por ser a versão mais recente disponível, quando do desenvolvimento deste trabalho. 

\subsection{OpenCV}
O OpenCV é um conjunto de bibliotecas desenvolvidas com o objetivo de provêr visão computacional em tempo real. Desenvolvida originalmente em C mudou sua plataforma para C++ com o objetivo de se tornar mais facil o desenvolvimento de novas funcionalidades, melhor implementação e aplicação de padrõee, com suporte em Python e Java (Android), com suporte em Windows, Linux, iOS, Mac OS e Android.
\subsection{Java \& JNI}
\subsection{Processo de Criptografia}
\subsection{Sistema de Criptografia Utilizando Imagens}

\section{Resultados}
\section{Conclusão}


\bibliographystyle{sbc}
\bibliography{sbc-template}

\end{document}
